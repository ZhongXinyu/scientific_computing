\documentclass[letterpaper,11pt]{article}
\usepackage{latexsym}
\usepackage[empty]{fullpage}
\usepackage{titlesec}
\usepackage{marvosym}
\usepackage[usenames,dvipsnames]{color}
\usepackage{verbatim}
\usepackage{enumitem}
\usepackage[hidelinks]{hyperref}
\usepackage{fancyhdr}
\usepackage[english]{babel}
\usepackage{tabularx}
\input{glyphtounicode}

%----------FONT OPTIONS----------
% sans-serif
% \usepackage[sfdefault]{FiraSans}
% \usepackage[sfdefault]{roboto}
% \usepackage[sfdefault]{noto-sans}
% \usepackage[default]{sourcesanspro}

% serif
% \usepackage{CormorantGaramond}
% \usepackage{charter}


\pagestyle{fancy}
\fancyhf{} % clear all header and footer fields
\fancyfoot{}
\renewcommand{\headrulewidth}{0pt}
\renewcommand{\footrulewidth}{0pt}

% Adjust margins
\addtolength{\oddsidemargin}{-0.5in}
\addtolength{\evensidemargin}{-0.5in}
\addtolength{\textwidth}{1in}
\addtolength{\topmargin}{-.5in}
\addtolength{\textheight}{1.0in}

\urlstyle{same}

\raggedbottom
\raggedright
\setlength{\tabcolsep}{0in}

% Sections formatting
\titleformat{\section}{
  \vspace{-4pt}\scshape\raggedright\large
}{}{0em}{}[\color{black}\titlerule \vspace{-5pt}]

% Ensure that generate pdf is machine readable/ATS parsable
\pdfgentounicode=1

%-------------------------
% Custom commands
\newcommand{\resumeItem}[1]{
  \item\small{
    {#1 \vspace{-2pt}}
  }
}

\newcommand{\resumeSubheading}[4]{
  \vspace{-2pt}\item
    \begin{tabular*}{0.97\textwidth}[t]{l@{\extracolsep{\fill}}r}
      \textbf{#1} & #2 \\
      \textit{\small#3} & \textit{\small #4} \\
    \end{tabular*}\vspace{-7pt}
}

\newcommand{\resumeSubSubheading}[2]{
    \item
    \begin{tabular*}{0.97\textwidth}{l@{\extracolsep{\fill}}r}
      \textit{\small#1} & \textit{\small #2} \\
    \end{tabular*}\vspace{-7pt}
}

\newcommand{\resumeProjectHeading}[2]{
    \item
    \begin{tabular*}{0.97\textwidth}{l@{\extracolsep{\fill}}r}
      \small#1 & #2 \\
    \end{tabular*}\vspace{-7pt}
}

\newcommand{\resumeSubItem}[1]{\resumeItem{#1}\vspace{-4pt}}

\renewcommand\labelitemii{$\vcenter{\hbox{\tiny$\bullet$}}$}

\newcommand{\resumeSubHeadingListStart}{\begin{itemize}[leftmargin=0.15in, label={}]}
\newcommand{\resumeSubHeadingListEnd}{\end{itemize}}
\newcommand{\resumeItemListStart}{\begin{itemize}}
\newcommand{\resumeItemListEnd}{\end{itemize}\vspace{-5pt}}

%-------------------------------------------
%%%%%%  RESUME STARTS HERE  %%%%%%%%%%%%%%%%%%%%%%%%%%%%


\begin{document}

%----------HEADING----------


\begin{center}
    \textbf{\Huge \scshape Xinyu Zhong} \\ \vspace
    \small 07536907294 $|$ xz447@cam.ac.uk $|$ 
    %\href{https://linkedin.com/in/x}{\underline{linkedin.com/in/x}} $|$
    %\href{https://github.com/x}{\underline{github.com/x}}
\end{center}


%-----------EDUCATION-----------
\section{Education}
  \resumeSubHeadingListStart
    \resumeSubheading
      {University of Cambridge}{ Cambridge, United Kingdom}
      {Bachelor of Art}{ Expected December 2023}
          \resumeItemListStart
            \resumeItem{Physics}
            \resumeItem{Math}
            \resumeItem{Materials science}
            
        \resumeItemListEnd

  \resumeSubHeadingListEnd

%-----------EXPERIENCE-----------
\section{Experience}
  \resumeSubHeadingListStart

    \resumeSubheading
      {Research on algorithm design}{ March 2019 - June 2019}
      {Ant Colony Algorithm}{ Beijing, China}
      \resumeItemListStart
        \resumeItem{Study and optimization of Ant Colony Algorithm on path planning.}
        \resumeItem{Improved the update of pheromone in ACO to reduce the chance of been traped in local optimal solution.}
        \resumeItem{Designed a new algorithm based on the original ACO called Partial Ant Colony Algorithm.}
        \resumeItem{Using Python test and applied the designed algorithm on path planning problenm.}
        \resumeItem{Published a paper on EEEP2019.}
      \resumeItemListEnd
            

    \resumeSubheading
      {Apply for Ross Mathematics program}{ January 2019 - February 2019}
      {Applicant}{ Beijing, China}
      \resumeItemListStart
        \resumeItem{Accomplished 4 big and complex problems, wrote a 60 pages solution.}
        \resumeItem{Topics include number theory, series and sequence, linear algebra, geometry, functions in complex field.}
        \resumeItem{Was accepted by the program, the acceptance rate was less than 10 percent.}
        \resumeItem{Using Python test and applied the designed algorithm on path planning problenm.}
        \resumeItem{Published a paper based on the result with one partner.}
      \resumeItemListEnd
    
    \resumeSubheading
      {Ross Mathematics program}{ June 2019 - August 2019}
      {Fresher}{ Ohio, United States}
      \resumeItemListStart
        \resumeItem{Devoted and profound study on number theory for 1.5 months.}
        \resumeItem{Studied Ring theory, Filed theory, Vector space, Group theory.}
        \resumeItem{Deeply studied all kinds of arithmetical functions.}
        \resumeItem{Deeply studied the properties of the residue system of primes.}
        \resumeItem{Proved the residue systems of primes are cyclic groups.}
        \resumeItem{Proved Fermat's two square theorem by using residue system of primes.}
        \resumeItem{Proved Fermat's two square theorem by using Minkowski's theorem and geometry.}
        \resumeItem{Proved Quadratic reciprocity.}
        \resumeItem{Introductory level to p-adic numbers.}
      \resumeItemListEnd
    
    
    
    \resumeSubHeadingListEnd

%-----------COMPITATIONS-----------
\section{Competitions}
  \resumeSubHeadingListStart

    \resumeSubheading
      {CAP High School Physics Prize Examination}{ April 2019}
      {Canadian Association of Physics}{ Beijing, China}
      \resumeItemListStart
        \resumeItem{Out Standing Award Globally}
        \resumeItem{Gold Award Nationally}
        \resumeItem{Top five in China}
      \resumeItemListEnd
    

    \resumeSubheading
      {2019 ASDAN Math Tournament}{ August 2019}
      {ASDAN China}{ Beijing, China}
      \resumeItemListStart
        \resumeItem{High Distinction in Algebra, Top 10 percent}
        \resumeItem{Top 10 in Geometry, Top 10}
      \resumeItemListEnd
    
    \resumeSubheading
      {Australian Mathematics Competition}{2018}
      {Australian Mathematics Trust}{ Beijing, China}
      \resumeItemListStart
        \resumeItem{Distinction level, Year eleven}
      \resumeItemListEnd
    
    \resumeSubheading
      {National Junior Electronic Engineer Championship}{ August 2016}
      {China Radio Sports Association}{ Chengdu, China}
      \resumeItemListStart
        \resumeItem{1 gold medal and 1 silver medal}
      \resumeItemListEnd
    
    
    \resumeSubHeadingListEnd

  
%-----------PROJECTS-----------
\section{Projects}

    \resumeSubHeadingListStart
      \resumeProjectHeading
          {\textbf{Simulation of refraction} $|$ \emph{Python, Optical geometry}}{July 2021 -- July 2021}
          \resumeItemListStart
            \resumeItem{Use Python and pyplot library to simulate the refraction of light}
            \resumeItem{Able to simulate any shape of lens and refractive index}
            \resumeItem{The light may com4 from any chosen direction}
            \resumeItem{Able to simulate refraction of light in a multiple layered lens}
            \resumeItem{The program was created as a library and can be easily imported to use}
          \resumeItemListEnd

      \resumeProjectHeading
          {\textbf{Matrix Calculation} $|$ \emph{Python, Linear algebra}}{October 2021 -- October 2021}
          \resumeItemListStart
            \resumeItem{Created a python library to carry out all arithmetic calculation on matrix}
            \resumeItem{Functions include but not limited to find inverse, multiplication, Hermitian, cominor.}
          \resumeItemListEnd
      
      \resumeProjectHeading
          {\textbf{Wave Dispersion} $|$ \emph{Python, Wave}}{November 2021 -- November 2021}
          \resumeItemListStart
            \resumeItem{Using python to simulate the dispersion of wave}
            \resumeItem{Output a GIF or video document to show the dispersion of waves as time goes on}
          \resumeItemListEnd

      \resumeProjectHeading
          {\textbf{Simulate Fraunhofer Diffraction Pattern} $|$ \emph{Python, Fraunhofer diffraction}}{November 2021 -- November 2021}
          \resumeItemListStart
            \resumeItem{Using python to simulate the diffraction pattern of plane wave with a given shape of aperture}
            \resumeItem{The result shows the intensity of light at the screen}
            \resumeItem{The intensity is represented by the depth of color}
            \resumeItem{An algorithm was designed to convert intensity to color}
            \resumeItem{An algorithm was designed to find the 2D Fourier transform}
          \resumeItemListEnd

      \resumeProjectHeading
          {\textbf{Factorial} $|$ \emph{Analytic continuation, Calculus}}{ February 2020 -- February 2020}
          \resumeItemListStart
            \resumeItem{Find a function of the analytic continuation of factorial function}
            \resumeItem{The domain of factorial function was expanded from integer to real number}
            \resumeItem{This function can be proved is essentially equivalent to Gamma function}
            \resumeItem{A detailed note about how it is derived and the source of my idea was taken done}
          \resumeItemListEnd

      \resumeProjectHeading
          {\textbf{Albumen Curve} $|$ \emph{Mathematics, Geometry}}{March 2020 -- March 2020}
          \resumeItemListStart
            \resumeItem{Define a type of curve called `Albumen curve'}
            \resumeItem{It was defined as rotating a length at certain angle around any closed curve}
            \resumeItem{The calculation of the area of these curves was studied, a general formula was summarized}
            \resumeItem{A detailed note about how it was derived was taken done}
          \resumeItemListEnd

      \resumeProjectHeading
          {\textbf{Find 24} $|$ \emph{Python, Probability}}{December 2020 -- December 2020}
          \resumeItemListStart
            \resumeItem{Wrote a Python library to play the classical poker game `calculate 24'}
            \resumeItem{The program can automatically give the expression of answer}
            \resumeItem{An algorithm with high efficiency was designed}
            \resumeItem{The program was than used to study the probability of `having solution' for all integers}
          \resumeItemListEnd

      \resumeProjectHeading
          {\textbf{Calculation of Root} $|$ \emph{Computational mathematics}}{March 2018 -- April 2018}
          \resumeItemListStart
            \resumeItem{Find an efficient algorithm to calculate $\sqrt[n]{m}$}
            \resumeItem{My partner and I found an iterative formula to approximate $\sqrt[n]{m}$}
            \resumeItem{The accuracy of this algorithm increase exponentially with the number of iterations}
            \resumeItem{The algorithm was tested by Python program and the result is satisfactory}
            \resumeItem{A complete note including the math derivation, source of our idea and test results was taken down}
          \resumeItemListEnd
         

    \resumeSubHeadingListEnd

%-----------TECHNICAL SKILLS-----------
\section{Technical Skills}
 \begin{itemize}[leftmargin=0.15in, label={}]
    \small{\item{
     Familiar with Python programming \\
     Familiar with Latex \\
     Familiar with multiple developing environments, include PyCharm, Spyder, JupiterNote, VS code \\
     Familiar with the use of Excle, Word Document and Power Point \\
     Familiar with multiple libraries in Python, include numpy, pyplot, pandas, and so on \\
     Be able to create and use my own python libraries\\
     Introductory level to DataBase and SQL language \\
     Knowledge of circuit design and welding techniques 
    }}
 \end{itemize}


%-------------------------------------------
\end{document}